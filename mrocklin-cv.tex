\documentclass[margin,line]{res}


\oddsidemargin -.5in
\evensidemargin -.5in
\textwidth=6.0in
\itemsep=0in
\parsep=0in

\newenvironment{list1}{
  \begin{list}{\ding{113}}{%
      \setlength{\itemsep}{0in}
      \setlength{\parsep}{0in} \setlength{\parskip}{0in}
      \setlength{\topsep}{0in} \setlength{\partopsep}{0in} 
      \setlength{\leftmargin}{0.17in}}}{\end{list}}
\newenvironment{list2}{
  \begin{list}{$\bullet$}{%
      \setlength{\itemsep}{0in}
      \setlength{\parsep}{0in} \setlength{\parskip}{0in}
      \setlength{\topsep}{0in} \setlength{\partopsep}{0in} 
      \setlength{\leftmargin}{0.2in}}}{\end{list}}

\begin{document}
\newcommand{\link}[1]{\texttt{#1}}

\name{Matthew D. Rocklin\vspace*{.1in}}

\begin{resume}
\section{\sc Contact Information}
\vspace{.05in}
%=========INFO=========
\begin{tabular}{@{}p{2in}p{4in}}
Department of Computer Science   &         \\ 
University of Chicago & {\it E-mail:}  mrocklin@cs.uchicago.edu\\       
Chicago, IL  60615 USA  & {\it webpage:} http://people.cs.uchicago.edu/\verb+~+mrocklin/ \\     
\end{tabular}


%========= RESEARCH =========
\section{\sc Research Interests}

I am interested in uncertainty quantification, numerical linear algebra, complex networks and the symbolic representation of high performance computations. 

Particular projects include accelerating statistical simulations using automatic derivatives and optimizing array-based programs for efficient allocation onto heterogeneous compute clusters. 

%========= EDUCATION ==========
\section{\sc Education}
{\bf University of Chicago}, Chicago, IL\\
%{\em Department of Statistics} 
\vspace*{-.1in}
\begin{list1}
\item[] Ph.D. Student, Computational Mathematics \hfill {\bf 2008-Present}
\item[] M.S. Computer Science \hfill {\bf 2011}
\end{list1}

{\bf University of California, Berkeley}, Berkeley, CA\\
%{\em Department of Mathematics and Statistics} 
\vspace*{-.1in}
\begin{list1}
\item[] B.A., Physics, Mathematics, and Astronomy \hfill {\bf May 2007}
\end{list1}

%========= HONORS ==========
\section{\sc Honors and Awards} 

McCormack Fellowship - University of Chicago\\
Departmental Teaching Assistant Award 2009-2010\\
Regents and Chancellors Scholarship - University of California, Berkeley\\

%========= TEACHING ==========
\section {\sc Teaching}

%\vspace{-.1cm}
{\bf Lecturer}
\begin{list1}
\item[] {\em 152 Introduction to Computer Science} \hfill {\bf Summer 2011}
\item[] {\em 152 Introduction to Computer Science} \hfill {\bf Summer 2012}
\end{list1}

{\bf Lab Instructor}
\begin{list1}
\item[] {\em 121-122 - CS with Applications}  \hfill {\bf Autumn-Winter 2009-10}
\item[] {\em 121-122 - CS with Applications}  \hfill {\bf Autumn-Winter 2010-11}
\item[] {\em 121-122 - CS with Applications}  \hfill {\bf Autumn-Winter 2011-12}
\item[] {\em 102 - Introduction to Web Applications}  \hfill {\bf Spring 2012}
\end{list1}

{\bf Teaching Assistant}
\begin{list1}
\item[] {\em 153 - Foundations of Software (theory)} \hfill {\bf Autumn 2008}
\item[] {\em 106 - Introduction to Computer Science (C++)} \hfill {\bf Winter 2009}
\item[] {\em 337 - Scientific Visualization} \hfill {\bf Spring 2009}
\end{list1}

I am also a mentor for Google Code-in and Google Summer of Code under the SymPy project. 

%========= PUBLICATIONS ==========
\section{\sc Publications}

{\bf Journal Papers}\\
M. Rocklin, A. Pinar, \textit{"On Clustering on Graphs with Multiple Edge Types
"} Internet Mathematics, 2012 (in press)

E. Constantinescu,V. Zavala, M. Rocklin, S. Lee, and M. Anitescu, \textit{"A Computational Framework for Uncertainty Quantification and Stochastic Optimization in Unit Commitment with Wind Power Generation."} IEEE Transactions on Power Systems, 2009


{\bf Conference Proceedings}\\
M. Rocklin, A. Pinar, \textit{"Latent Clustering on Graphs with Multiple Edge Types"} Algorithms and Models for the Web-Graph, 2011

M. Rocklin, A. Pinar, \textit{"Computing an Aggregate Edge-Weight Function for Clustering Graphs with Multiple Edge Types."} Algorithms and Models for the Web-Graph, 2010

{\bf Other}\\
M. Rocklin, A. Terrel, \textit{Symbolic Statistics using SymPy} Computations in
Science and Engineering, 2012 (in press)

M. Rocklin \textit{"Uncertainty Quantification and Sensitivity Analysis in Dynamical Systems"}, Masters Thesis, May, 2011

Patent 7620209: \textit{“Method and apparatus for dynamic space-time imaging system”}

%========= WORK EXPERIENCE ==========
\section{\sc Professional Experience}

%\vspace{-.3cm}
{\bf Sandia National Laboratory } - Livermore, CA\\
{\em Summer researcher} \hfill {\bf Summer 2010}\\
Clustering on Graphs with multiple Similarity Metrics 

%\vspace{-.3cm}
{\bf Argonne National Laboratory} - Chicago, IL\\
{\em Givens Fellow} \hfill {\bf Summer 2009}\\
Uncertainty Quantification and Sensitivity Analysis of Numerical Weather Prediction Models.

%\vspace{-.3cm}
{\bf UC Berkeley Physics Department} - Berkeley, CA\\
{\em Staff Research Assistant} \hfill {\bf 2007 - 2008}\\
Algorithms and software to probabilistically track intracellular movement of vesicles moving within the bodies of plant cells. Developed biophysics educational tools

%\vspace{-.3cm}
{\bf National Solar Observatory} - Tucson, AZ\\
{\em Student Research Assistant} \hfill {\bf Summer 2006}\\
Numerical simulations and data analysis tools for design of Advanced Technology Solar Telescope.

%\vspace{-.3cm}
{\bf Berkeley Engineering and Research/4D Imaging} - Berkeley, CA\\
{\em Developer} \hfill {\bf 2003 - 2005}\\
3d scanner based on structured light techniques. Began startup engineering company. Initial sole developer of a project which eventually grew to become an independent and profitable company

{\bf Open Source }\\
Active contributor to the SymPy project, symbolic modeling in Python.

%=========== CODE ================
\section{\sc Code}

I endeavor to produce high-level code (symbolics) to address low-level (numeric) efficiency concerns. A substantial fraction of my work is available at \link{http://www.github.com/mrocklin/}.

\begin{center}
\begin{tabular}{|l | c| }
\hline
Topic  & Experience  \\
\hline
\hline
Languages & Python, C,  C++, C\#, Java, MatLab \\
&  CUDA, IDL, LabVIEW, LaTeX \\
&  SQL, Maude, Clojure \\
\hline
Version Control & Git, SVN \\
\hline
Preferred Development Stack & Linux, C, Python, Git, Latex\\
\hline
\end{tabular}
\end{center}

\end{resume}
\end{document}




